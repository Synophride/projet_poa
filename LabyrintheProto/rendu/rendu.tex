\documentclass{article}
\usepackage[utf8]{inputenc}
\usepackage[french]{babel}
\usepackage[margin=2cm]{geometry} 
\usepackage{tikz}
\usepackage{setspace}
\usepackage{multicol}
\usepackage{graphicx}
\usepackage{float}


\title{Projet de programmation objet avancée}
\author{Aurélien Terrain et Julien Guyot}
\date{4 Mai 2019}


\begin{document}
\maketitle
\section{Introduction}
Ce projet porte sur la réalisation à partir d'une base
d'un jeu de type «kill them all»
: Le but est de tuer un ensemble de «gardiens»
pour accéder à un trésor.


Durant ce rapport, nous décrirons brièvement la structure du code
ainsi que les choix faits, avant de conclure.


Par ailleurs, nous avons généré une documentation basique (voire
rudimentaire) via Doxygen, décrivant toutes les fonctions et
tous les attributs rajoutés aux diverses classes , et qui
est disponible dans les répertoires {\verb docs/html/index.html }
et {\verb docs/latex/refman.pdf }


\section{Structure du code}
Le code que nous avons produit est disponible dans les trois classes
suivantes, comportant chacune un fichier {\verb .cc } et {\verb .h }.

\subsection{ Labyrinthe }
Le principal choix fait a été de modifier la structure du tableau
{\verb _data[][] } : chaque case peut contenir une valeur comprise
entre 0 ({\verb EMPTY }) et 5 ( {\verb TREASURE } ), indiquant sur la
nature de ce qui occupe la case.


Une fonctionnalité que nous avons implémenté a été de permettre que
les boîtes soient destructibles pour permettre au joueur de regenérer
sa santé.


\subsection{ Gardien }
Les gardiens sont dotés d'un certain nombre de points de vie au début
du jeu, et perdent leur précision quand ils se font toucher par une
boule de feu.
On remarque aussi qu'ils possèdent une perte de
précision de base, contrairement au chasseur.


Leur décision est faite en fonction d'un potentiel de protection,
calculé par rapport à des paramètres, plus un quota aléatoire. Le
garde ne fait pas de décision à chaque appel de $update()$, mais prend
une décision, et la suit pendant un temps aléatoire (mais borné).


Les décisions peuvent être d'attaquer (auquel cas le garde se dirige
vers le joueur), de défendre (aller vers le trésor) ou explorer
(exploration au hasard dans le labyrinthe).




Nous pouvons aussi remarquer que
la fonction de mouvement
utilisée par les gardien est relativement rustique, et conduit
souvent à un amoncellement de gardiens sur une
case donnée.

\subsection{ Chasseur }
Le chasseur dispose d'un plus grand nombre de points de vie que les
gardiens, et perd moins de précision.

\section{ Conclusion }
Nous avons implémenté un jeu fonctionnel à partir du code donné en
cours. 
\end{document}
